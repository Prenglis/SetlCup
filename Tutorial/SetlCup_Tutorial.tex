%% Based on a TeXnicCenter-Template by Tino Weinkauf.
%%%%%%%%%%%%%%%%%%%%%%%%%%%%%%%%%%%%%%%%%%%%%%%%%%%%%%%%%%%%%

%%%%%%%%%%%%%%%%%%%%%%%%%%%%%%%%%%%%%%%%%%%%%%%%%%%%%%%%%%%%%
%% HEADER
%%%%%%%%%%%%%%%%%%%%%%%%%%%%%%%%%%%%%%%%%%%%%%%%%%%%%%%%%%%%%
\documentclass[a4paper,10pt]{article}
% Alternative Options:
%	Paper Size: a4paper / a5paper / b5paper / letterpaper / legalpaper / executivepaper
% Duplex: oneside / twoside
% Base Font Size: 10pt / 11pt / 12pt


%% Language %%%%%%%%%%%%%%%%%%%%%%%%%%%%%%%%%%%%%%%%%%%%%%%%%
\usepackage[utf8]{inputenc} %francais, polish, spanish, ...
\usepackage[T1]{fontenc}
%\usepackage[ansinew]{inputenc}
%
\usepackage{lmodern} %Type1-font for non-english texts and characters
\usepackage[pdfborder={0 0 0}]{hyperref}

%% Packages for Graphics & Figures %%%%%%%%%%%%%%%%%%%%%%%%%%
\usepackage{graphicx} %%For loading graphic files
%\usepackage{subfig} %%Subfigures inside a figure
%\usepackage{pst-all} %%PSTricks - not useable with pdfLaTeX
\usepackage{verbatim}
\usepackage{listings}
\usepackage{fancyvrb}
%% Please note:
%% Images can be included using \includegraphics{Dateiname}
%% resp. using the dialog in the Insert menu.
%% 
%% The mode "LaTeX => PDF" allows the following formats:
%%   .jpg  .png  .pdf  .mps
%% 
%% The modes "LaTeX => DVI", "LaTeX => PS" und "LaTeX => PS => PDF"
%% allow the following formats:
%%   .eps  .ps  .bmp  .pict  .pntg


%% Math Packages %%%%%%%%%%%%%%%%%%%%%%%%%%%%%%%%%%%%%%%%%%%%
\usepackage{amsmath}
\usepackage{amsthm}
\usepackage{amsfonts}


%% Line Spacing %%%%%%%%%%%%%%%%%%%%%%%%%%%%%%%%%%%%%%%%%%%%%
%\usepackage{setspace}
%\singlespacing        %% 1-spacing (default)
%\onehalfspacing       %% 1,5-spacing
%\doublespacing        %% 2-spacing


%% Other Packages %%%%%%%%%%%%%%%%%%%%%%%%%%%%%%%%%%%%%%%%%%%
%\usepackage{a4wide} %%Smaller margins = more text per page.
%\usepackage{fancyhdr} %%Fancy headings
%\usepackage{longtable} %%For tables, that exceed one page


%%%%%%%%%%%%%%%%%%%%%%%%%%%%%%%%%%%%%%%%%%%%%%%%%%%%%%%%%%%%%
%% Remarks
%%%%%%%%%%%%%%%%%%%%%%%%%%%%%%%%%%%%%%%%%%%%%%%%%%%%%%%%%%%%%
%
% TODO:
% 1. Edit the used packages and their options (see above).
% 2. If you want, add a BibTeX-File to the project
%    (e.g., 'literature.bib').
% 3. Happy TeXing!
%
%%%%%%%%%%%%%%%%%%%%%%%%%%%%%%%%%%%%%%%%%%%%%%%%%%%%%%%%%%%%%

%%%%%%%%%%%%%%%%%%%%%%%%%%%%%%%%%%%%%%%%%%%%%%%%%%%%%%%%%%%%%
%% Options / Modifications
%%%%%%%%%%%%%%%%%%%%%%%%%%%%%%%%%%%%%%%%%%%%%%%%%%%%%%%%%%%%%

%\input{options} %You need a file 'options.tex' for this
%% ==> TeXnicCenter supplies some possible option files
%% ==> with its templates (File | New from Template...).
\usepackage[section]{placeins}
\makeatletter
\AtBeginDocument{%
  \expandafter\renewcommand\expandafter\subsection\expandafter{%
    \expandafter\@fb@secFB\subsection
  }%
}
\makeatother
\newcommand{\mRefFigure}[1]{(siehe Figur \ref{#1}, Seite \pageref{#1})}
\newcommand{\mRefTable}[1]{(siehe Tabelle \ref{#1}, Seite \pageref{#1})}
\newcommand{\mRefEqu}[1]{(siehe Gleichung \ref{#1}, Seite \pageref{#1})}

%%%%%%%%%%%%%%%%%%%%%%%%%%%%%%%%%%%%%%%%%%%%%%%%%%%%%%%%%%%%%
%% DOCUMENT
%%%%%%%%%%%%%%%%%%%%%%%%%%%%%%%%%%%%%%%%%%%%%%%%%%%%%%%%%%%%%
\begin{document}

\pagestyle{empty} %No headings for the first pages.


%% Title Page %%%%%%%%%%%%%%%%%%%%%%%%%%%%%%%%%%%%%%%%%%%%%%%
%% ==> Write your text here or include other files.

%% The simple version:
\title{SetlCup Tutorial}
\author{Jonas Eilers}
%\date{} %%If commented, the current date is used.
\maketitle

%% The nice version:
%\input{titlepage} %%You need a file 'titlepage.tex' for this.
%% ==> TeXnicCenter supplies a possible titlepage file
%% ==> with its templates (File | New from Template...).


%% Inhaltsverzeichnis %%%%%%%%%%%%%%%%%%%%%%%%%%%%%%%%%%%%%%%
\newpage
\tableofcontents %Table of contents
\cleardoublepage %The first chapter should start on an odd page.

\pagestyle{plain} %Now display headings: headings / fancy / ...



%% Chapters %%%%%%%%%%%%%%%%%%%%%%%%%%%%%%%%%%%%%%%%%%%%%%%%%
%% ==> Write your text here or include other files.

%\input{intro} %You need a file 'intro.tex' for this.


%%%%%%%%%%%%%%%%%%%%%%%%%%%%%%%%%%%%%%%%%%%%%%%%%%%%%%%%%%%%%
%% ==> Some hints are following:

%\chapter{Some small hints}\label{hints}
%
%\section{German Umlauts and other Language Specific Characters}\label{umlauts}
%You can type german umlauts like '�', '�', or '�' directly in this file.
%This is also true for other language specific characters like '�', '�' etc.
%
%There are problems with automatic hyphenation when using language
%specific characters and OT1-encoded fonts. In this case, use a
%T1-encoded Type1-font like the Latin Modern font family (\verb#\usepackage{lmodern}#).
%
%
%\section{References}\label{references}
%Using the commands \verb#\label{name}# and \verb#\ref{name}# you are able
%to use references in your document. Advantage: You do not need to think
%about numerations, because \LaTeX\ is doing that for you.
%
%For example, in section \ref{dividing} on page \pageref{dividing} hints for
%dividing large documents are given.
%
%Certainly, references do also work for tables, figures, formulas\ldots
%
%Please notice, that \LaTeX\ usually needs more than one run (mostly 2) to
%resolve those references correctly.
%
%
%\section{Dividing Large Documents}\label{dividing}
%You can divide your \LaTeX-Document into an arbitrary number of \TeX-Files
%to avoid too big and therefore unhandy files (e.g. one file for every chapter).
%
%For this, you insert in your main file (this one) for every subfile
%the command '\verb#\input{subfile}#'. This leads to the same behavior
%as if the content of the subfile would be at the place of the \verb#\input#-Command.

%% <== End of hints
%%%%%%%%%%%%%%%%%%%%%%%%%%%%%%%%%%%%%%%%%%%%%%%%%%%%%%%%%%%%%

\chapter{Functionality}

The Setlx-addition SetlCup is a LR-Parser-Generator based on JavaCup.
The idea is to use a user given scanner- and parser-definition and create an AST out of a given input using the definitions.

In this document the needed syntax of the definitions is examined and the given output is evaluated.

A sample input file is divided into three Sections:
\begin{enumerate}
	\item Commentpart
	\item Scanner-Part
	\item Parser-Part
\end{enumerate}

\section{Comment-Part}
In the comment-part everything which is written will not be used by the Program itself. It is adviced to comment your idea behind the parser and scanner structure in this section.
The section is ended with the "\%\%\%" symbol.

\section{Scanner-Part}
The scanner is responsible for checking whether the input file consists of the defined tokens. It can be written like this:
\begin{lstlisting}[frame=single,numbers=left,basicstyle=\footnotesize]
INTEGER := 1-9[0-9]*|0;
ASTERISK :=  \*;
WHITESPACE := [ ];
SKIP := ASTERISK | INTEGER | WHITESPACE;
\end{lstlisting}
\begin{enumerate}
	\item In line 1 the Token "INTEGER" is defined. Tokens are in the following way:\\
					token\_name := regex ;
	\item Predefined tokens in Regular Expressions like "$*,+,?,|,\{,\},(,),\cdots$" need to be escaped.
	\item In some contexts tokens like Whitespaces are not needed. They can be skipped by using defining the "SKIP"-Token with the tokens, which shall be skipped. Multiple tokens need to be seperated by a pipe "|".
\end{enumerate}

\section{Parser-Part}
In this part the grammar-rules are defined with the following syntax:
\begin{lstlisting}[frame=single,numbers=left,basicstyle=\footnotesize]
rule_name := rule_element:id {: action_code :} 
	| rule_element
	| {: action_code :} 
	| ;
\end{lstlisting}
\begin{itemize}
	\item[rule\_name] The rule\_name is the name of the rule. It is possible to reference defined rules via their rule\_name
	\item[rule\_element] The element can consist of multiple Tokens (defined in the scanner) and rule\_names. Each can have an id, which is possible to be used in the action\_code.
	\item[action\_code] The action\_code is an optional part in a rule. It needs to be at the end of the rule it self. Each rule\_element can have an action\_code. In this action\_code Setlx Code can be written. By using the variable "`result"' it is possible to pass values between rules. The id of the elements in the respective rule can be referred to by using its name.
	\item[|] The pipe seperates the rule\_elements.
\end{itemize}
\newpage
\section{Example}
\lstinputlisting[frame=single,numbers=left,basicstyle=\footnotesize]{math_expression_grammar_ast.g}
\lstinputlisting[frame=single,numbers=left,basicstyle=\footnotesize]{math_expression_input.txt}
\lstinputlisting[frame=single,numbers=left,basicstyle=\footnotesize]{math_expression_input.txt}

\lstinputlisting[frame=single,numbers=left,basicstyle=\footnotesize, extendedchars=true]{factorial_output.txt}

%%%%%%%%%%%%%%%%%%%%%%%%%%%%%%%%%%%%%%%%%%%%%%%%%%%%%%%%%%%%%
%% BIBLIOGRAPHY AND OTHER LISTS
%%%%%%%%%%%%%%%%%%%%%%%%%%%%%%%%%%%%%%%%%%%%%%%%%%%%%%%%%%%%%
%% A small distance to the other stuff in the table of contents (toc)
\addtocontents{toc}{\protect\vspace*{\baselineskip}}

%% The Bibliography
%% ==> You need a file 'literature.bib' for this.
%% ==> You need to run BibTeX for this (Project | Properties... | Uses BibTeX)
%\addcontentsline{toc}{chapter}{Bibliography} %'Bibliography' into toc
%\nocite{*} %Even non-cited BibTeX-Entries will be shown.
\newpage
\bibliographystyle{alpha} %Style of Bibliography: plain / apalike / amsalpha / ...
\bibliography{cs} %You need a file 'literature.bib' for this.



%%%%%%%%%%%%%%%%%%%%%%%%%%%%%%%%%%%%%%%%%%%%%%%%%%%%%%%%%%%%%
%% APPENDICES
%%%%%%%%%%%%%%%%%%%%%%%%%%%%%%%%%%%%%%%%%%%%%%%%%%%%%%%%%%%%%
%\appendix
%% ==> Write your text here or include other files.

%\input{FileName} %You need a file 'FileName.tex' for this.


\end{document}

