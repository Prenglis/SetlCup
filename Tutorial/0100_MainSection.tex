\chapter{Functionality}

The Setlx-addition SetlCup is a LR-Parser-Generator based on JavaCup.
The idea is to use a user given scanner- and parser-definition and create an AST out of a given input using the definitions.

In this document the needed syntax of the definitions is examined and the given output is evaluated.

A sample input file is divided into three Sections:
\begin{enumerate}
	\item Commentpart
	\item Scanner-Part
	\item Parser-Part
\end{enumerate}

\section{Comment-Part}
In the comment-part everything which is written will not be used by the Program itself. It is adviced to comment your idea behind the parser and scanner structure in this section.
The section is ended with the "\%\%\%" symbol.

\section{Scanner-Part}
The scanner is responsible for checking whether the input file consists of the defined tokens. It can be written like this:
\begin{lstlisting}[frame=single,numbers=left,basicstyle=\footnotesize]
INTEGER := 1-9[0-9]*|0;
ASTERISK :=  \*;
WHITESPACE := [ ];
SKIP := ASTERISK | INTEGER | WHITESPACE;
\end{lstlisting}
\begin{enumerate}
	\item In line 1 the Token "INTEGER" is defined. Tokens are in the following way:\\
					token\_name := regex ;
	\item Predefined tokens in Regular Expressions like "$*,+,?,|,\{,\},(,),\cdots$" need to be escaped.
	\item In some contexts tokens like Whitespaces are not needed. They can be skipped by using defining the "SKIP"-Token with the tokens, which shall be skipped. Multiple tokens need to be seperated by a pipe "|".
\end{enumerate}

\section{Parser-Part}
In this part the grammar-rules are defined with the following syntax:
\begin{lstlisting}[frame=single,numbers=left,basicstyle=\footnotesize]
rule_name := rule_element:id {: action_code :} 
	| rule_element
	| {: action_code :} 
	| ;
\end{lstlisting}
\begin{itemize}
	\item[rule\_name] The rule\_name is the name of the rule. It is possible to reference defined rules via their rule\_name
	\item[rule\_element] The element can consist of multiple Tokens (defined in the scanner) and rule\_names. Each can have an id, which is possible to be used in the action\_code.
	\item[action\_code] The action\_code is an optional part in a rule. It needs to be at the end of the rule it self. Each rule\_element can have an action\_code. In this action\_code Setlx Code can be written. By using the variable "`result"' it is possible to pass values between rules. The id of the elements in the respective rule can be referred to by using its name.
	\item[|] The pipe seperates the rule\_elements.
\end{itemize}
\newpage
\section{Example}
\lstinputlisting[frame=single,numbers=left,basicstyle=\footnotesize]{math_expression_grammar_ast.g}
\lstinputlisting[frame=single,numbers=left,basicstyle=\footnotesize]{math_expression_input.txt}
\lstinputlisting[frame=single,numbers=left,basicstyle=\footnotesize]{math_expression_input.txt}

\lstinputlisting[frame=single,numbers=left,basicstyle=\footnotesize, extendedchars=true]{factorial_output.txt}